\renewcommand{\thepage}{Copyright}
\begin{center}
\textcopyright\,Copyright 2022
\\
Tyler Chen
\thispagestyle{empty}
\end{center}


\clearpage
\renewcommand{\thepage}{UW title page}
\begin{center}

\textbf{Lanczos-based  methods for matrix functions}

\vspace{3em}
Tyler Chen 

\vfill

\textit{A dissertation}
\\
submitted in partial fulfillment of the
\\
requirements for the degree of

\vspace{3em}
\textbf{Doctor of Philosophy}

\vspace{3em}
University of Washington
\\
2022

\vspace{3em}
Reading Committee:
\\Anne Greenbaum, Chair
\\Thomas Trogdon, Chair
\\Aleksandr Aravkin

\vfill
Program Authorized to Offer Degree:
\\
Applied Mathematics

\end{center}

\thispagestyle{empty}

\clearpage
\renewcommand{\thepage}{UW abstract}
\begin{center}
University of Washington

\vspace{3em}
    \textit{Abstract}

\vspace{3em}
\textbf{Lanczos-based  methods for matrix functions}

\vspace{3em}
Tyler Chen

\vspace{3em}

Chairs of the Supervisory Committee:
\\Anne Greenbaum
\\Thomas Trogdon

\vspace{3em}
Department of Applied Mathematics

\end{center}

\vfill
{\fontsize{11}{24}\selectfont
We study Lanczos-based methods for tasks involving matrix functions.
We begin by resurfacing a range of ideas regarding matrix-free quadrature which, to the best of our knowledge, have not been treated simultaneously. 
This enables the development of a unified perspective from which a number of commonly used randomized methods for spectrum and spectral sum approximation can be understood.
We proceed to develop optimal Krylov subspace methods for approximating the product of a rational matrix function with a fixed vector.
Finally, we show how the optimality of such methods can be used to obtain fine-grained spectrum dependent bounds for standard Lanczos-based methods for approximating a wide class of matrix functions applied to a vector.
}
\thispagestyle{empty}
