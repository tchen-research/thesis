

\section{A posteriori bounds for Gaussian quadrature}

The a priori bounds from the previous section do not take into account any properties of \( \Psi \) beyond the smallest and largest points of the support. 
Thus, they are unable to take advantage of any fine-grained properties which may be present in \( \Psi \) (for instance, due to clustered eigenvalues of \( \vec{A} \)).
In this section, prove an interseting property of Gaussian quadrature rules which allows us to obtain a posteriori bounds.
In practice, these bounds are often better than the a priori bounds from the previous section \cite{chen_trogdon_ubaru_21}.


\begin{definition}
A function \( \gamma \) has a sign change at \( x \) if there exists \( x' < x \) such that \( \gamma(x')\neq 0 \) and \( x = \inf\{t > x' : \gamma(t) \gamma(x') < 0 \} \).
\end{definition}

\begin{restatable}{lemma}{weaksign}
\label{thm:weak_sign}
Suppose \( \gamma \) is weakly increasing on an interval \( (a,b) \).
Then \( \gamma \) and has a sign change at \( x \) if and only if there exists \( x'<x \) such that \( \gamma(x') < 0 \), \( \gamma(y) \leq 0 \) for all \( y\in(a,x) \) and \( \gamma(y) > 0 \) for all \( y\in(x,b) \).
%The \( \gamma \) has at most one sign change on \( (a,b) \).
\end{restatable}

\begin{proof}
Suppose that \( \gamma \) has a sign change at \( x \) and let \( x' \) be as in the definition.
Then, for any \( t > x' \), since \( \gamma \) is weakly increasing, \( \gamma(x') \leq \gamma(t) \).
This means that \( \gamma(t) \gamma(x') < 0 \) implies \( \gamma(x') < 0 \) and \( \gamma(t) > 0 \).
Since \( x = \inf\{t > x' : \gamma(t) \gamma(x') < 0 \}\), we have \( \gamma(y) > 0 \) for all \( y\in(x,b) \) and \( \gamma(y) \leq 0 \) for all \( y\in(x',x) \) and \( \gamma(x') < 0 \)

The reverse direction follows directly from the definition of sign change.
\end{proof}

\begin{theorem}[{\cite[Theorem 22.1]{karlin_shapley_72}}]
\label{thm:moments_CDF}
Suppose \( \mu \) and \( \nu \) are two probability distribution functions constant on the complement of \( [a,b] \) whose moments are equal up to degree \( s \).
Define \( \gamma : [a,b] \to [0,1] \) by \( \gamma(x) = \mu(x) - \nu(x) \).
Then \( \gamma \) is identically zero or changes sign at least \( s \) times.
\end{theorem}

\begin{proof}
Suppose \( \gamma \neq 0 \).
For the sake of contradiction, suppose also that \( \gamma \) has fewer than \( s \) sign changes.
Then, there exists a degree at most \( s-1 \) polynomial \( r \) such that for all \( x \in [a,b] \), \( r(x) \gamma(x) \geq 0 \); i.e. pick \( r \) to have a sign change at every sign change of \( \gamma \).

Thus, since \( \gamma \neq 0 \) is right continuous and \( r \) is continuous, 
\begin{equation*}
    \int_{a}^{b} r(x) \gamma(x) \d{x} > 0.
\end{equation*}

Let \( R \) be an antiderivative of \( r \).
Then, by integrating by parts over the closed interval \( [a,b] \),
\begin{equation*}
    \int_{a}^{b} r(x) \gamma(x) \d{x} 
    = \left[ R(x) \gamma(x) \right]_{a^-}^b - \int_{a}^{b} R(x) \d\gamma(x).
\end{equation*}
Since \( \mu \) and \( \nu \) are equal on the compliment of \( [a,b] \),
\begin{equation*}
    \left[ R(x) \gamma(x) \right]_a^b 
    = R(b) ( \mu(b) - \nu(b)) - R(a) (\mu(a^-) - \nu(a^-))
    = 0
\end{equation*}
and, since \( \mu \) and \( \nu \) share moments up to degree \( s \),
\begin{equation*}
    \int_{a}^{b} R(x) \d\gamma(x)
    = \int_{a}^{b} R(x) (\d\mu(x) - \d\nu(x))
    =0 .
\end{equation*}

This contradicts the earlier assertion that this integral is non-zero.%, so \( \gamma \) must be zero or have at least \( k-1 \) sign changes.
\end{proof}

Note that for a probability distribution function, \( \qq[g]{\mu}{2k-1} \) is piecewise constant with \( k \) points of discontinuity.
Using the fact that \( \qq[g]{\mu}{2k-1} \) and \( \mu \) share moments up to degree \( 2k-1 \) along with \cref{thm:moments_CDF}, we immediatley obtain the following bounds on \( \qq[g]{\mu}{2k-1} \) (proved in \cref{sec:proofs} for completeness).

\note{check indexing}

\begin{restatable}{corollary}{gqinterlace}
\label{thm:gq_interlace}
Suppose \( \mu \) is a probability distribution function constant on the complement of \( [a,b] \). % with finite moments up to degree \( 2k-1 \).
Let \( \{ \theta_j \}_{j=1}^{k} \) and \( \{ \omega_j \}_{j=1}^{k} \) respectively be the nodes and weights of the Guassian quadrature rule \( \qq[g]{\mu}{2k-1} \).
Define \( \qq[\downarrow]{\mu}{2k-1} \) and \( \qq[\uparrow]{\mu}{2k-1} \) by
\begin{equation*}
    \qq[\downarrow]{\mu}{2k-1}(x)
    := \sum_{j=1}^{k-1} \omega_{j} \bOne\left[ \theta_{j+1} \leq x \right]
    ,\qquad\text{and}\qquad
    \qq[\uparrow]{\mu}{2k-1}(x)
    := \omega_1 + \sum_{j=2}^{k} \omega_{j} \bOne\left[\theta_{j-1} \leq x \right].
\end{equation*}
    Then, for all \( x \in [a,b] \),
\begin{equation*}
    \qq[\downarrow]{\mu}{2k-1}(x)
    \leq \mu(x) 
    \leq \qq[\uparrow]{\mu}{2k-1}(x).
\end{equation*}
\end{restatable}


\begin{proof}
%    If \( \qq[g]{\mu}{s} = \mu \) the statement is trivially true.
    Suppose \( \qq[g]{\mu}{2k-1} \neq \mu \) and define \( \gamma(x) = \mu(x) - \qq[g]{\mu}{2k-1}(x) \).
    Observe that for any \( j = 1,\ldots, k-1 \), \( \qq[g]{\mu}{2k-1} \) is constant on \( (\theta_j,\theta_{j+1}) \), so \( \gamma \) is weakly increasing on this interval.
    \Cref{thm:weak_sign} states that if \( \gamma \) change signs at some point \( y_j \in (\theta_j,\theta_{j+1}) \) then \( \gamma(x) > 0 \) for all \( x\in (y_j,\theta_{j+1}) \), \( j=1,\ldots, k-1 \) and \( \gamma(x) \leq 0 \) for all \( x\in(\theta_j,y_j) \), so \( \gamma \) cannot change signs at any other point in \( (\theta_j,\theta_{j+1}) \).
     
    Observe further that on \( (a,\theta_1) \), \( \qq[g]{\mu}{2k-1}(x) = 0 \leq \mu(x) \) so \( \gamma(x) \geq 0 \), and on \( (\theta_n,b) \), \( \qq[g]{\mu}{2k-1}(x) = 1 \geq \mu(x) \) so \( \gamma(x) \leq 0 \).
    Thus, by \cref{thm:weak_sign}, no sign changes can occur on these intervals.%\( (a,\theta_1) \) or \( (\theta_n,b) \).

    As a result, the only possible locations for sign changes of \( \gamma \) on \( (a,b) \) are \( \{\theta_j\}_{j=1}^{k} \) and \( \{y_j\}_{j=1}^{k-1} \).
    This is exactly \( 2k-1 \) possible sign changes, so by \cref{thm:moments_CDF}, a sign change must occur at each of these points.
    In particular, 
    \iffalse
    Using this and \cref{thm:weak_sign}, for all \( j=1,\ldots, k-1 \),
    \begin{equation*}
        \mu(x) &> \qq[g]{\mu}{2k-1}(x), & x&\in(y_j,\theta_{j+1})
        \\
        \mu(x) &\leq \qq[g]{\mu}{2k-1}(x) , & x&\in(\theta_j,y_j)
    \end{equation*}
    and additionally, \fi
    since \( \gamma \) has a sign change at \( \theta_j \),
    \begin{equation*}
        \qq[g]{\mu}{2k-1}(\theta_j^-) 
        \leq \mu(\theta_j) \leq \qq[g]{\theta_j}{s}.
    \end{equation*}
    Therefore, for \( x\in (\theta_j,\theta_{j+1}) \),
    \begin{equation*}
        \qq[\downarrow]{\mu}{2k-1}(x) =
        \qq[g]{\mu}{2k-1}(\theta_{j}^-) \leq
        \mu(\theta_j) \leq
        \mu(x) 
        \leq \mu(\theta_{j+1})
        \leq \qq[g]{\mu}{2k-1}(\theta_{j+1})
        = \qq[\uparrow]{\mu}{2k-1}(x).
        \tag*{\qedhere}
    \end{equation*}
\end{proof}


In turn, \cref{thm:gq_interlace} implies bounds on the Wasserstein and Kolmogorov--Smirnov distances between \( \mu \) and \( \qq[g]{\mu}{2k-1} \).


\begin{restatable}{corollary}{gqapost}
\label{thm:gq_apost}
Suppose \( \mu \) is a probability distribution function constant on the complement of \( [a,b] \). % with finite moments up to degree \( 2k-1 \).
Let \( \{ \theta_j \}_{j=1}^{k} \) and \( \{ \omega_j \}_{j=1}^{k} \) respectively be the nodes and weights of the Guassian quadrature rule \( \qq[g]{\mu}{2k-1} \).
Then
\begin{align*}
    \KS(\mu,\qq[g]{\mu}{2k-1}) 
    &\leq \max_{j=1,\ldots,k} \omega_j 
    \\
    \W(\mu,\qq[g]{\mu}{2k-1})
    &\leq \sum_{j=0}^{k} \max\{\omega_j,\omega_{j+1}\} ( \theta_{j+1} - \theta_j )
\end{align*}
where we define \( \theta_0 = a \), \( \theta_{k+1} = b \), and \( d_0 = d_{k+1} = 0 \).
\end{restatable}


\begin{proof}
Note that
\begin{align*}
    | \mu(x) - \qq[g]{\mu}{k}(x) | 
    &\leq \max\{ | \qq[g]{\mu}{2k-1}(x) - \qq[\uparrow]{\mu}{2k-1}(x) |, |\qq[g]{\mu}{2k-1}(x) - \qq[\downarrow]{\mu}{2k-1}(x) | \} 
    \\&= \max\{ \omega_j,\omega_{j+1} \} \bOne[ x\in [\theta_j,\theta_{j+1}) ].
\end{align*}
Thus,
\begin{equation*}
    \KS(\mu,\qq[g]{\mu}{2k-1})
    = \sup_{x} | \mu(x) - \qq[g]{\mu}{k}(x) |
    \leq \max_{j=1,\ldots, k} d_j. 
\end{equation*}
and
\begin{equation*}
    \W(\mu,\qq[g]{\mu}{2k-1})
    = \int_{a}^{b} |\mu(s) - \qq[g]{\mu}{2k-1}(s) | \d{s}
    %&\leq \int_{a}^{b} \max\{ | \qq[g]{\mu}{2k-1}(s) - \gqupper{k}(\mu)(s) |, |\qq[g]{\mu}{2k-1}(s) - \gqlower{k}(\mu)(s) | \} \d{s}
    \leq \sum_{j=0}^{k} \max\{\omega_j,\omega_{j+1}\} ( \theta_{j+1} - \theta_j ).
    \tag*{\qedhere}
\end{equation*}
\end{proof}



